\chapter{Conclusions and Future Work}

This thesis presented a fully-automated solver for Mixed-Bag jigsaw puzzles.  This solver outperforms the current state of the art both in terms of solution quality and the maximum number of puzzles it can solve simultaneously.  What is more, unlike the state of the art, it requires no externally supplied information beyond the set of puzzle pieces.

Opportunities currently exist to improve the Mixed-Bag Solver's performance.  First, the quality ceiling of solved outputs is significantly affected by the assembler.  This solver was designed to be largely independent of the assembler used, meaning the solver's performance will improve as better solvers are proposed. As such, an improved assembler that uses multiple best buddies to prioritize placement is currently under development.  What is more, this new assembler will also address some of the performance limitations of Paikin~\& Tal's algorithm for images with low best buddy density.

Another area where future investigation is planned is in the determination of the threshold for hierarchical clustering, which is currently set at a fixed value.  It is expected that a more dynamic approach may improve the clustering overall.

One additional aspect of the solver's performance that could be improved is in the selection of the stitching pieces.  As explained in Section~\ref{sec:stitchingPieceSelection}, a stitching piece is by definition a member of a saved segment.  In some cases, the mini-assembly may not actually expand the segment, which would prevent segment clustering.  As such, segment stitching may be improve if pieces that belong to no segment are also selected for stitching since these pieces may be more likely to bridge gaps between segments, in particular for images with low best buddy density.