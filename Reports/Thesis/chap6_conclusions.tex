\chapter{Conclusions and Future Work}

This thesis presented a fully-automated solver for Mixed-Bag jigsaw puzzles.  The solver outperforms the current state of the art both in terms of solution quality and also the maximum number of puzzles it can solve simultaneously.  What is more, unlike the state of the art, it requires no externally supplied information beyond the set of puzzle pieces.

Opportunities exist to improve the Mixed-Bag Solver's performance.  First, the assembler places a ceiling on the quality of the solver outputs.  This solver is largely independent of the assembler used, meaning the solver's performance will improve as better assemblers are proposed. As such, an improved assembler that uses multiple best buddies to prioritize placement is currently under development.  What is more, this new assembler addresses some of the performance limitations of Paikin~\& Tal's algorithm for images with low best buddy density.

In addition, the threshold for hierarchical clustering is currently set to a fixed value.  It is expected that a dynamic approach may improve the clustering overall and in turn the solver's performance.

Lastly, it was explained in Section~\ref{sec:stitchingPieceSelection} that stitching pieces are always members of saved segments.  In some cases, the mini-assembly may not actually expand the segment, which would prevent segment clustering.  As such, stitching may improve if pieces not assigned to a segment are also used since these pieces may be more likely to bridge inter-segment gaps.