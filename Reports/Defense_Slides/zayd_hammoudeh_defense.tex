\documentclass[10pt]{beamer}
\usetheme[
%%% options passed to the outer theme
%    hidetitle,           % hide the (short) title in the sidebar
     hideauthor,          % hide the (short) author in the sidebar
%    hideinstitute,       % hide the (short) institute in the bottom of the sidebar
%    shownavsym,          % show the navigation symbols
%    width=2cm,           % width of the sidebar (default is 2 cm)
%    hideothersubsections,% hide all subsections but the subsections in the current section
%    hideallsubsections,  % hide all subsections
    left               % right of left position of sidebar (default is right)
%%% options passed to the color theme
%    lightheaderbg,       % use a light header background
]{AAUsidebar}

% If you want to change the colors of the various elements in the theme, edit and uncomment the following lines
% Change the bar and sidebar colors:
\definecolor{spartanBlue}{RGB}{0,85,162}
\definecolor{spartanYellow}{RGB}{229,168,35}
\definecolor{spartanGray}{RGB}{147,149,151}
\definecolor{darkGreen}{RGB}{34, 139, 34}

\setbeamercolor{frametitle}{use=structure,fg=white,bg=spartanBlue}
\setbeamercolor{title}{fg=frametitle.bg}

\setbeamercolor{header}{bg=spartanBlue}

\setbeamercolor{title in sidebar}{fg=frametitle.bg} % Headings on the side bar
\setbeamercolor{palette sidebar secondary}{fg=frametitle.bg} % Headings on the side bar
%\setbeamercolor{palette sidebar primary}{fg=frametitle.bg} % Slide titles (i.e., not slide sections)

\setbeamercolor{section in toc}{use=palette sidebar secondary,fg=palette sidebar secondary.fg} 
\setbeamercolor{subsection in toc}{use=normal text,fg=palette normal text.fg} 
\setbeamercolor{block title}{fg=spartanBlue} % Color of headings
\setbeamercolor{local structure}{fg=spartanBlue} % Color of enumerated items (e.g. lists, bullets, etc.)
\setbeamercolor{bibliography item}{use=local structure,fg=local structure.fg} % Color number of bibliography items
\setbeamercolor{bibliography entry author}{use=normal text,fg=normal text.fg} % Color bibliography entry.  Includes not just author but all fields


\usepackage{environ}
\usepackage{tikz}
\usepackage{algorithm}
\usepackage[noend]{algpseudocode}
\usepackage{listings} % Used for printing source code in papers
\usepackage{pgfplots} % Used to create graphs
\usepackage{makecell} % Used to create thick links in tables
\usepackage{algorithm} % Used for writing algorithms in a paper
\usepackage[noend]{algpseudocode} % Allows psuedocode keywords (e.g., "if", "while", "for", etc.) in algorithms.
\usepackage[utf8]{inputenc}
\usepackage[english]{babel}
\usepackage[T1]{fontenc}
% Or whatever. Note that the encoding and the font should match. If T1
% does not look nice, try deleting the line with the fontenc.
\usepackage{helvet}

% colored hyperlinks
\newcommand{\chref}[2]{%
  \href{#1}{{\usebeamercolor[bg]{AAUsidebar}#2}}%
}

\title[A Fully-Automated Solver for Multiple Square Jigsaw Puzzles Using Hierarchical Clustering]% optional, use only with long paper titles
{\textbf{A Fully-Automated Solver for Multiple Square Jigsaw Puzzles Using Hierarchical Clustering}}

\subtitle{}  % could also be a conference name

\date{October 27, 2016}

\author[Zayd Hammoudeh] % optional, use only with lots of authors
{
    Zayd Hammoudeh\\
    \href{mailto:hammoudeh@gmail.com}{{\tt hammoudeh@gmail.com}}
}
% - Give the names in the same order as they appear in the paper.
% - Use the \inst{?} command only if the authors have different
%   affiliation. See the beamer manual for an example

\institute[
%  {\includegraphics[scale=0.2]{aau_segl}}\\ %insert a company, department or university logo
    Dept.\ of Computer Science\\
    San Jose State University\\
] % optional - is placed in the bottom of the sidebar on every slide
{% is placed on the title page
    Department of Computer Science\\
    San Jose State University\\
  
  %there must be an empty line above this line - otherwise some unwanted space is added between the university and the country (I do not know why;( )
}


% specify a logo on the titlepage (you can specify additional logos an include them in 
% institute command below
\pgfdeclareimage[height=1.5cm]{titlepagelogo}{images/SJSU_Seal_Blue} % placed on the title page
\titlegraphic{% is placed on the bottom of the title page
  \pgfuseimage{titlepagelogo}
%  \hspace{1cm}\pgfuseimage{titlepagelogo2}
}


%%%%%%%%%%%%%%%%%%%%%%%%%%%%%%%%%%%%%%%%%%
%% Handles Slide Numbering For Appendix %%
%%%%%%%%%%%%%%%%%%%%%%%%%%%%%%%%%%%%%%%%%%
\newcommand{\backupbegin}{
   \newcounter{finalframe}
   \setcounter{finalframe}{\value{framenumber}}
}
\newcommand{\backupend}{
   \setcounter{framenumber}{\value{finalframe}}
}
%%%%%%%%%%%%%%%%%%%%%%%%%%%%%%%%%%%%%%%%%%


%%%%%%%%%%%%%%%%%%%%%%%%%%%%%%%%%%%%%%%%%%
%%   Scales tikz images to side size    %%
%%%%%%%%%%%%%%%%%%%%%%%%%%%%%%%%%%%%%%%%%%
\makeatletter
\newsavebox{\measure@tikzpicture}
\NewEnviron{scaletikzpicturetowidth}[1]{%
  \def\tikz@width{#1}%
  \def\tikzscale{1}\begin{lrbox}{\measure@tikzpicture}%
  \BODY
  \end{lrbox}%
  \pgfmathparse{#1/\wd\measure@tikzpicture}%
  \edef\tikzscale{\pgfmathresult}%
  \BODY
}
\makeatother
%%%%%%%%%%%%%%%%%%%%%%%%%%%%%%%%%%%%%%%%%%



\begin{document}
% the titlepage
{
\begin{frame}[plain,noframenumbering] % the plain option removes the sidebar and header from the title page
\titlepage
\end{frame}}
%%%%%%%%%%%%%%%%



%% TOC
%\begin{frame}{Agenda}{}
%	\tableofcontents
%\end{frame}
%%%%%%%%%%%%%%%%%



\section{Introduction}
\begin{frame}{Introduction}{Jigsaw Puzzles}
  \begin{itemize}
    \item First jigsaw puzzle introduced in the 1760s.  Modern jigsaw puzzles were introduced in the 1930s.
    \vfill
    \item First computation jigsaw puzzle solver introduced in 1964.
    \vfill
    \item Solving a jigsaw puzzle is NP Complete \cite{altman1990, demaine2007}
    \vfill
    \item<2-> \textbf{Example Applications:} DNA fragment reassembly, shredded document reconstruction, speech descrambling, and image editing.
    \begin{itemize}
        \item<3-> In most cases, the original, ``ground-truth'' image is unknown.
    \end{itemize}
  \end{itemize}
\end{frame}
%%%%%%%%%%%%%%%%


\begin{frame}{Introduction}{Jig Swap Puzzles}
    \textbf{Jigswap Puzzles} -- Variant of the traditional jig saw puzzle
    \begin{itemize}
        \item All pieces are equal-sized squares
        \item Substantially more difficult
    \end{itemize}
    \vfill
    % Delay the table on the slide
    \begin{tabular}{ >{\centering\arraybackslash}m{0.45\textwidth} >{\centering\arraybackslash}m{0.45\textwidth} }
	\onslide<2->{\fbox{\includegraphics[width=0.365\textwidth]{./images/muffins_300x200.jpg}}} & \onslide<3->{\fbox{\includegraphics[width=0.365\textwidth]{./images/muffins_scrambled.jpg}}}
	\\ ~\\
	\onslide<2->{Ground-Truth Image} & \onslide<3->{Randomized Jig Swap Puzzle}
	\\ ~\\
  \end{tabular}
\end{frame}
%%%%%%%%%%%%%%%%



\begin{frame}{Introduction}{Jig Swap Puzzle Types}
\onslide<2->{There are four primary jig swap puzzle types as formalized by~\cite{gallagher2012}. In all cases, the ``ground-truth'' input(s) is unknown.} 
    \vfill
    \begin{itemize}
        \item<2-> \textbf{Type~1}: Puzzle dimension and piece rotation are known.  One or more ``anchor'' pieces are fixed in their correct location.
        \vfill
        \item<3-> \textbf{Type~2}: All piece locations and rotations unknown.  Puzzle dimensions may be known.
        \vfill
        \item<4-> \textbf{Type~3}: All piece locations are known.  Only rotation is unknown.
        \vfill
        \item<5-> \textbf{Mixed-Bag}: Pieces come from multiple puzzles.
    \end{itemize}
    \vfill
    \onslide<6->{Mixed-Bag Puzzles are the focus of this thesis.}
\end{frame}
%%%%%%%%%%%%%%%%



\begin{frame}{Introduction}{Best Buddies}
    \begin{itemize}
        \item \textbf{Basis of all Modern Jig Swap Solvers}: The more compatible two pieces are on their respective sides, the more likely they are to be adjacent.
        \vfill
        \item \textbf{Best Buddies}: Any pair of puzzles pieces that are more compatible with each other on their respective sides than they are to any other piece~\cite{pomeranz2011}
        \vfill
\begin{equation} \label{eq:pomeranzBestBuddyDefinition}
\centering
\begin{split}
	\begin{matrix}
		\forall{p_{k}}\forall{s_z},C(p_i, s_x, p_j, s_y) \geq C(p_i, s_x, p_k, s_z)
		\\
		\\
		\textnormal{and}
		\\
		\\
		\forall{p_{k}}\forall{s_z},C(p_j, s_y, p_i, s_x) \geq C(p_j, s_y, p_k, s_z)
	\end{matrix}
\end{split}
\end{equation} 
        \vfill
        \item \textbf{Importance of Best Buddies}: Strong indicator of piece adjacency
    \end{itemize}
\end{frame}
%%%%%%%%%%%%%%%%



\subsection{Previous Work}
\begin{frame}{Previous Work}{}
    \begin{itemize}
        \onslide<1->{\item \textbf{Cho \textit{et al.}}~\cite{cho2010} -- Introduced the first Modern Jig Swap Puzzle Solver Introduced
        \begin{itemize}
            \setlength\itemsep{.8em}
            \item Graphical model-based Type~1 solver
            \item Puzzle dimensions are known
            \item Used one or more anchor pieces
            \item Established the standard test conditions including puzzle piece size and image encoding using the LAB~colorspace
        \end{itemize}}
        \vfill
        \onslide<2->{\item \textbf{Pomeranz \textit{et al.}}~\cite{pomeranz2011} -- Iterative, greedy jigsaw Type~1 puzzle solver
        \begin{itemize}
            \setlength\itemsep{.8em}
            \item Eliminated the use of anchor pieces 
            \item Created multiple solver benchmarks of various sizes
            \item Introduced the concept of ``best buddies''
        \end{itemize}}
    \end{itemize}
\end{frame}
%%%%%%%%%%%%%%%%



\section{Mixed-Bag Solver}
\begin{frame}{Mixed-Bag Solver}{}
  \begin{itemize}
    \item First jigsaw puzzle introduced in the 1760s.  Modern jigsaw puzzles were introduced in the 1930s.
    \vfill
    \item First computation jigsaw puzzle solver introduced in 1964.
    \vfill
    \item Solving a jigsaw puzzle is NP Complete \cite{altman1990, demaine2007}
    \vfill
    \item<2-> \textbf{Example Applications:} DNA fragment reassembly, shredded document reconstruction, speech descrambling, and image editing.
    \begin{itemize}
        \item<3-> In most cases, the original, ``ground-truth'' image is unknown.
    \end{itemize}
  \end{itemize}
\end{frame}

\subsection{Assembler}
\subsection{Segmentation}
\subsection{Stitching}
\subsection{Hierarchical Clustering}
\subsection{Final Seed Piece Selection}
\subsection{Final Assembly}

\section{Quantifying Quality}
\subsection{Direct Accuracy}
\subsection{Neighbor Accuracy}

\section{Experimental Results}
\begin{frame}{Experimental Results}{Input Puzzle Count}
  \begin{block}{Windows with MiKTeX}
    Apparently, MiKTeX does not include a local latex directory tree by default. Therefore, you first have to create it.
    \begin{enumerate}
      \item To do this, create a folder {\tt <somewhere>} named, e.g., {\tt texmf}
      \item Add this folder in the Roots tab of the MiKTeX Settings dialog
      \item Place the {\tt <dirstruct>} in your newly created local latex directory tree\\
    {\tt <somewhere>\textbackslash texmf}\\
      \item Open the MiKTeX Settings dialog and click Refresh FNDB.
    \end{enumerate}
  \end{block}
\end{frame}
%%%%%%%%%%%%%%%%



\subsection{Input Puzzle Count}
\begin{frame}{Experimental Results}{Determining Input Puzzle Count}
  \begin{block}{Windows with MiKTeX}
    Apparently, MiKTeX does not include a local latex directory tree by default. Therefore, you first have to create it.
    \begin{enumerate}
      \item To do this, create a folder {\tt <somewhere>} named, e.g., {\tt texmf}
      \item Add this folder in the Roots tab of the MiKTeX Settings dialog
      \item Place the {\tt <dirstruct>} in your newly created local latex directory tree\\
    {\tt <somewhere>\textbackslash texmf}\\
      \item Open the MiKTeX Settings dialog and click Refresh FNDB.
    \end{enumerate}
  \end{block}
\end{frame}
%%%%%%%%%%%%%%%%



\begin{frame}{Determining Input Puzzle Count}{Single Input Puzzle}
  \begin{block}{Windows with MiKTeX}
    Apparently, MiKTeX does not include a local latex directory tree by default. Therefore, you first have to create it.
    \begin{enumerate}
      \item To do this, create a folder {\tt <somewhere>} named, e.g., {\tt texmf}
      \item Add this folder in the Roots tab of the MiKTeX Settings dialog
      \item Place the {\tt <dirstruct>} in your newly created local latex directory tree\\
    {\tt <somewhere>\textbackslash texmf}\\
      \item Open the MiKTeX Settings dialog and click Refresh FNDB.
    \end{enumerate}
  \end{block}
\end{frame}
%%%%%%%%%%%%%%%%



\begin{frame}{Determining Input Puzzle Count}{Multiple Input Puzzles}
     \begin{center}
       \begin{scaletikzpicturetowidth}{0.9\textwidth}
        \begin{tikzpicture}[scale=\tikzscale]
          \begin{axis}[
                ybar, axis on top,
                height=8cm, width=12cm,
            	axis background/.style={fill=gray!10},
                bar width=0.4cm,
                ymajorgrids, tick align=inside,
                major grid style={draw=white},
                enlarge y limits={value=.1,upper},
                ymin=0, ymax=100,
                axis x line*=bottom,
                axis y line*=left,
                y axis line style={opacity=1},
                tickwidth=0pt,
                title=\textbf{Mixed-Bag Solver's Input Puzzle Count Error Frequency},
                enlarge x limits=0.2,
                legend style={
                    at={(0.5,-0.2)},
                    anchor=north,
                    legend columns=-1,
                    /tikz/every even column/.append style={column sep=0.5cm}
                },
                xlabel={Size of Input Puzzle Count Error},
                ylabel={Frequency (\%)},
                symbolic x coords={
                   0, 1, 2, 3},
               xtick=data,
               nodes near coords={
                \pgfmathprintnumber[precision=0]{\pgfplotspointmeta}
               }
            ]
        \addplot [fill=blue!30]
            coordinates {(0,74.5) (1,16.4) (2,7.3) (3,1.8)};
        \addplot [fill=red!30]
            coordinates {(0,44) (1,48) (2,4) (3,4)};
        \addplot [fill=green!30]
            coordinates {(0,50) (1,50) (2,0) (3,0)};
        \addplot [fill=gray]
            coordinates {(0,60) (1,20) (2,20) (3,0)};
        \legend{2 Puzzles, 3 Puzzles, 4 Puzzles, 5 Puzzles}
        \end{axis}
        \end{tikzpicture}
      \end{scaletikzpicturetowidth}  
    \end{center}
\end{frame}
%%%%%%%%%%%%%%%%



\subsection{Solver Comparison}
\begin{frame}{Experimental Results}{}
  \begin{block}{Windows with MiKTeX}
  \end{block}
\end{frame}
%%%%%%%%%%%%%%%%



\begin{frame}{Comparison of Solver Performance}{Shifted Enhanced Direct Accuracy Score (SEDAS)}
     \begin{center}
       \begin{scaletikzpicturetowidth}{0.9\textwidth}
        \begin{tikzpicture}[scale=\tikzscale]
          \begin{axis}[
            height=8cm, width=12cm,
            axis background/.style={fill=gray!10},
            title=\textbf{Effect of the Number of Input Puzzles on SEDAS},
            xlabel={\textbf{Number of Input Puzzles}},
            ylabel={\textbf{SEDAS}},
            xmin=1.5, xmax=5.5,
            ymin=0, ymax=1,
            xtick={2, 3, 4, 5},
            ytick={0,0.2,0.4,0.6,0.8,1.0},
            ymajorgrids=true,
            grid style=dashed,
            legend style={ at={(0.5,-0.25)}, anchor=north, legend columns=-1, /tikz/every even column/.append style={column sep=0.5cm}}
            ]
        \addplot [color=blue,mark=*,mark options={fill=blue},ultra thick]
            coordinates {(2,0.849835) (3,0.953583)
                 (4,0.88068) (5,0.792796)};
        \addplot [color=red,mark=square*,mark options={fill=red},ultra thick]
            coordinates {(2,0.757159) (3,0.799822)
                 (4,0.777987) (5,0.782815)};
        \addplot [color=darkGreen,mark=triangle*,mark options={fill=darkGreen},ultra thick]
            coordinates {(2,0.321232) (3,0.202879)
                 (4,0.108857) (5,0.09866)};
        \legend{MBS Correct Puzzle Count, MBS All, Paikin \& Tal}
        \end{axis}
        \end{tikzpicture}
      \end{scaletikzpicturetowidth}  
    \end{center}
\end{frame}
%%%%%%%%%%%%%%%%



\begin{frame}{Comparison of Solver Performance}{Enhanced Neighbor Accuracy Score (ENAS)}
     \begin{center}
       \begin{scaletikzpicturetowidth}{0.9\textwidth}
        \begin{tikzpicture}[scale=\tikzscale]
          \begin{axis}[
            height=8cm, width=12cm,
            axis background/.style={fill=gray!10},
            title=\textbf{Effect of the Number of Input Puzzles on ENAS},
            xlabel={\textbf{Number of Input Puzzles}},
            ylabel={\textbf{ENAS}},
            xmin=1.5, xmax=5.5,
            ymin=0, ymax=1,
            xtick={2, 3, 4, 5},
            ytick={0,0.2,0.4,0.6,0.8,1.0},
            ymajorgrids=true,
            grid style=dashed,
            legend style={ at={(0.5,-0.25)}, anchor=north, legend columns=-1, /tikz/every even column/.append style={column sep=0.5cm}}
            ]
			\addplot [color=blue,mark=*,mark options={fill=blue},ultra thick]
	coordinates {(2,0.932805) (3,0.955051)
		 (4,0.919987) (5,0.868454)};
			\addplot [color=red,mark=square*,mark options={fill=red},ultra thick]
	coordinates {(2,0.874472) (3,0.868832)
		 (4,0.862183) (5,0.876654)};
			\addplot [color=darkGreen,mark=triangle*,mark options={fill=darkGreen},ultra thick]
	coordinates {(2,0.462006) (3,0.364242)
		 (4,0.259996) (5,0.204337)};
        \legend{MBS Correct Puzzle Count, MBS All, Paikin \& Tal}
        \end{axis}
        \end{tikzpicture}
      \end{scaletikzpicturetowidth}  
    \end{center}
\end{frame}
%%%%%%%%%%%%%%%%



\begin{frame}{Comparison of Solver Performance}{Perfect Reconstruction}
     \begin{center}
       \begin{scaletikzpicturetowidth}{0.9\textwidth}
        \begin{tikzpicture}[scale=\tikzscale]
          \begin{axis}[
            height=8cm, width=12cm,
            axis background/.style={fill=gray!10},
            title=\textbf{Effect of the Number of Input Puzzles on ENAS},
            xlabel={\textbf{Number of Input Puzzles}},
            ylabel={\textbf{Perfect Reconstruction (\%)}},
            xmin=1.5, xmax=5.5,
            ymin=0, ymax=30,
            xtick={2, 3, 4, 5},
            ytick={0,5,10,15,20,25,30},
            ymajorgrids=true,
            grid style=dashed,
            legend style={ at={(0.5,-0.25)}, anchor=north, legend columns=-1, /tikz/every even column/.append style={column sep=0.5cm}}
            ]
			\addplot [color=blue,mark=*,mark options={fill=blue},ultra thick]
	coordinates {(2,29.3) (3,18.5)
		 (4,25.0) (5,20.0)};
			\addplot [color=red,mark=square*,mark options={fill=red},ultra thick]
	coordinates {(2,23.6) (3,18.8)
		 (4,15.6) (5,24.0)};
			\addplot [color=darkGreen,mark=triangle*,mark options={fill=darkGreen},ultra thick]
	coordinates {(2,5.5) (3,1.4)
		 (4,0) (5,0)};
        \legend{MBS Correct Puzzle Count, MBS All, Paikin \& Tal}
        \end{axis}
        \end{tikzpicture}
      \end{scaletikzpicturetowidth}  
    \end{center}
\end{frame}
%%%%%%%%%%%%%%%%


\subsection{Ten Puzzle Results}
\begin{frame}{Experimental Results}{Solving More than Five Puzzles}
  \begin{itemize}
	\item<1-> As the number of puzzles increase, the difficulty of simultaneously reconstructing the puzzles increases.
	\vfill  
  	\item<2-> \textbf{Current State of the Art} -- Paikin~\& Tal~\cite{paikin2015} solved up to five puzzles simultaneously.
	\vfill
  	\item<3-> \textbf{Current State of the Art} -- Paikin~\& Tal~\cite{paikin2015} solved up to five puzzles simultaneously.
  \end{itemize}
\end{frame}
%%%%%%%%%%%%%%%%


\begin{frame}{Experimental Results}{Ten Puzzle Results}
  	
  	\begin{itemize}
  		\item<1-> \textbf{Paikin~\& Tal}
  		\begin{itemize}
			\item Seed of nine images came from just three input images
			\item SEDAS and EDAS greater than 0.9 for only one image
			\item No perfectly reconstructed images
		\end{itemize}
		\vfill
  		
  		\item<2-> \textbf{Mixed-Bag Solver}
		\begin{itemize}
			\item SEDAS and EDAS greater than 0.9 for all images
			\item Four images perfectly reconstructed
			\item Results comparable to Paikin~\& Tal's algorithm solving each puzzle individually
		\end{itemize}	
		\vfill
  		\item<3-> \textbf{Conclusion}: Mixed-Bag Solver significantly outperforms Paikin~\& Tal's algorithm.	
  \end{itemize}
  
  		
  

\end{frame}
%%%%%%%%%%%%%%%%



\begin{frame}{Solver Quality Comparison}{Ten Puzzle Result}
  \begin{block}{Windows with MiKTeX}
  \end{block}
\end{frame}
%%%%%%%%%%%%%%%%

\section{Conclusions}
\begin{frame}{Conclusions}{}
  \begin{block}{Windows with MiKTeX}
  \end{block}
\end{frame}
%%%%%%%%%%%%%%%%


\subsection{Future Work}
\begin{frame}{Conclusions}{Future Work}
  \begin{enumerate}
    \item<1-> Improved Assembler
    \begin{enumerate}
    	\item<1-> Prioritize placement using multiple best buddies
    	\vfill
    	\item<2-> Address placement performance in regions with low best buddy density
    \end{enumerate}
    \vfill
    \item<2-> Dynamic determination of the segment clustering threshold
    \vfill
    \item<3-> Expanded stitching piece selection
    \vfill
  \end{enumerate}
\end{frame}
%%%%%%%%%%%%%%%%

\section*{}  % Close the previous section
\begin{frame}[t,allowframebreaks]{List of References}
	\bibliographystyle{ieeetr}
	\bibliography{references}
\end{frame}

\appendix
\backupbegin

\begin{frame}[t,allowframebreaks]{}

\end{frame}
\backupend


\end{document}
